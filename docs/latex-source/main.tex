%
% File hicss.tex
%
% Contact: Holm Smidt, hicss@hawaii.edu
%%
%%
%% Based on the style files for ACL 2015 by 
%% car@ir.hit.edu.cn, gdzhou@suda.edu.cn


\documentclass[10pt,sigconf, review]{article}
% \documentclass[sigconf,review]{acmart}

\input{hicss-packages.tex}
\newcommand{\sansserifformat}[1]{\fontfamily{cmss}{ #1}}%

\setlength\titlebox{6cm}

% You can expand the titlebox if you need extra space
% to show all the authors. Please do not make the titlebox
% smaller than 5cm (the original size).


\title{ \huge Secure Parcel Counter Using Object Detection}

 % Comment this for initial manuscript 
 % Uncomment this for final manuscript

\author{BD Zwane \\
  \textbf{<student number>}\\
  \em{Computer System Engineering}\\
  \em{Thswane University of Technology}\\
  \em{Pretoria South Africa}\\

  \And

  SH Khoza \\
  \textbf{214651459}\\
  \em{Computer System Engineering}\\
  \em{Thswane University of Technology}\\
  \em{Pretoria South Africa}\\

  \And 

  <another member> \\
  \textbf{<student number>}\\
  \em{Computer System Engineering}\\
  \em{Thswane University of Technology}\\
  \em{Pretoria South Africa}\\}

\date{}

\begin{document}

\maketitle

\begin{abstract}
A Parcel Counter is a domestic counter-to-counter parcel service where a client
can send or receive parcels from any where across the country. Due to high 
volume of parcel being sent at a given time, client parcel end up getting lost 
and some being received by wrong client. We develop a shape detection to add 
a level of protection to clients parcel. The implententaion of object detection
make more sence to parcel counter since the will be no nesesary to create an
account when you wnat to send your parcel and improve it will limit the humanly
mistale that are made when parkaging the parcel,i\\ 

Raspberry pi, Finite state machine, Multiplexers, detection algorithm
name, OpenCV

\end{abstract}

\section{Introduction}

Write this part in  paragraph
The first paragraph gives a general idea of the current world and how lives has
been affected by the new technology and methods of detection. (Please read some
document on shape recognition in security and give at least two reference that
are accessible so that I might be able to check your reference) when you
reference a document cite it as [1] and if the reference section [1] should be
the reference you tagged as [1]\\

The second paragraph should indicate what other people have done in the field
of security recognition. What are the  similar projects and please write this
as reference (you need to give 4 to 6 reference ) Please write this as though
you are telling the story.\\

The third paragraph should emphasize on the method that you have decided and
why you have decided to used is. (if you have different methods for each
methods you have 1 paragraph) you can support you choice with 1 or 2
reference\\

The fourth paragrath should tell us about the benefits of this project, what
are we gaining in implementing this project.\\

The last paragraph now is about the organization of the Report :  (Leave it as
it is as I am writing it for you ) : The remainder of this report is organized
as follow: Section two discuss the background and related work on “ Fill in the
name of your project”. Section 3 shown the methods used for this project.
Section 4: Shows the simulation obtained while section 5 discuss the obtained
results. Finally section 6 conclude the report.


\section{Background : related work}

First paragraph talk again about your project and its impact in our modern society.\\

Choose 2 or 3 projects that you know, that have been are close to your project
and explain why those projects are close to yours and how are they important.\\ 

The last paragraph of this section tells why your project is good and why is it
different with other projects.\\

\section{Methods}

This parts speaks about the methods that you will use to implement the project.
The methods for us will be mostly Finite sate machine, multiplexer, shape
detection methods. For each methods that you decide to use, present them in
paragraph

\subsection{Operations of the project}

Explain all the algorithm that you are using here. List them as A, B C … (2 or
3 algorithms are enough.) Algorithm. Flow chart of your project can be put
here. You can use any software to draw it and copy the image here ( Please
don’t draw it with words )


\subsection{Algorithms}
Explain all the algorithm that you are using here. List them as A, B C … (2
algorithms are enough.) Algorithm/\\
Flow chart should be used to explain your algorithm.\\
After explaining the algorithm, you can summarize it as follow:\\

Algorithm 1:
\begin{enumerate}
	\item Start the project
	\item Upload the image 
	\item Check if the shape is recognized ?
  \item Does it match with any stored or acknowledge shape
  \item Give the output message
\end{enumerate}

This is just an example that I have made for you. You can customize it as you
want.

\subsection{Equations}
If you have equations in your report please type them using the recommendation below
The equations are an exception to the prescribed specifications of this
template. You will need to determine whether or not your equation should be
typed using either the Times New Roman or the Symbol font (please no other
font).


\section{Simualtions}
Here you can put figures or images of your software simulations, the outputs
waves of Quartus can be placed and explain each figure. Put some pictures of
your projects simulations, some LCD display output.

% For one-column wide figures use
\begin{figure}[thb]
	% Use the relevant command to insert your figure file.
	% For example, with the graphicx package use
    \centering
	\includegraphics[trim={3cm 3cm 3cm 3cm}, clip,width=0.9\linewidth]{sample-image}
	% figure caption is below the figure
	\caption{Sample figure with caption.}
	\label{fig: sample-figure}       % Give a unique label
\end{figure}


Author names and affiliations must be included in the submitted Final Paper for
Publication. Leave two 12-point blank lines after the author’s information. 

\section{Second and following pages}
\label{sect:pdf}

The second and following pages should begin 1.0 inch (2.54 cm) from the top
edge. On all pages, the bottom margin should be 1-1/8 inches (2.86 cm) from the
bottom edge of the page for 8.5 x 11-inch paper. (Letter-size paper)

\section{Type-style and fonts}
\label{sec:type-style}

Please note that {\em Times New Roman} is the preferred font for the text of
you paper. \textbf{If you must use another font}, the following are considered
base fonts.  You are encouraged to limit your font selections to Helvetica,
Arial, and Symbol as needed. These fonts are automatically installed with the
viewing software. 



 


\section{Discussions}
Discuss all the results of your projects Discuss what you have achieve and tell
us why it is important. Explain every output figure that you have. What it does
and how you understand it.

\section{Conclusion}
Conclude your paper efficiently by giving the overall statement of your
project. State what you have achieved and why it was important achieving. If
possible give 


% \section{References} 

% The template will number citations consecutively within brackets [1]. The
% sentence punctuation follows the bracket [2]. Refer simply to the reference
% number, as in [3]—do not use “Ref. [3]” or “reference [3]” except at the
% beginning of a sentence: “Reference [3] was the first ...”

% \begin{small}
%     [1] G. Eason, B. Noble, and I. N. Sneddon, “On certain integrals of Lipschitz-Hankel type involving products of Bessel functions,” Phil. Trans. Roy. Soc. London, vol. A247, pp. 529–551, April 1955. (references)

%     [2] J. Clerk Maxwell, A Treatise on Electricity and Magnetism, 3rd ed., vol. 2. Oxford: Clarendon, 1892, pp.68–73.
% \end{small}


\bibliographystyle{ieeetr}
\bibliography{sample}


\end{document}
